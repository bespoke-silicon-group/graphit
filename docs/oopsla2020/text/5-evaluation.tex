% !TEX root = paper.tex

\section{Evaluation} \label{sec:eval}

\subsection{Experimental Setup}

\subsection{Host Side Coordination}

\begin{figure}[t]
	\centering
	\subfloat[][bfs]{\includegraphics[width=0.3\textwidth]{direction-speedup.pdf}\label{direction-bfs}}
	\subfloat[][cc]{\includegraphics[width=0.3\textwidth]{direction-speedup.pdf}\label{direction-cc}}
	\subfloat[][pr-nibble]{\includegraphics[width=0.3\textwidth]{direction-speedup.pdf}\label{direction-pr}}
	\caption{This figure shows the speedup for various graph traversal directions on several different input graphs. Speedup is show over the sparse push direction.}
	\label{fig:direction-plot}
\end{figure}

\subsection{Device Side Coordination}

\subsection{Parallel Model Comparison}

\begin{figure*}[t]
	\centering
	\subfloat[][bfs]{\includegraphics[width=0.3\textwidth]{speedup_comp.pdf}\label{speedup-bfs}}
	\subfloat[][cc]{\includegraphics[width=0.3\textwidth]{speedup_comp.pdf}\label{speedup-cc}}
	\subfloat[][pr-nibble]{\includegraphics[width=0.3\textwidth]{speedup_comp.pdf}\label{speedup-pr}}
	\label{speedup-methods}
    \caption{Speedup comparison between the host side and device side coordination parallel execution models. Figure shows speedup over increasing number of active threads on three different graph benchmarks.}
\end{figure*}

\begin{figure}[t]
	\centering
	\subfloat[][bfs]{\includegraphics[width=0.3\textwidth]{energy_comp.pdf}\label{energy-bfs}}
	\subfloat[][cc]{\includegraphics[width=0.3\textwidth]{energy_comp.pdf}\label{energy-cc}}
	\subfloat[][pr-nibble]{\includegraphics[width=0.3\textwidth]{energy_comp.pdf}\label{energy-pr}}
	\caption{Energy comparison between the host side and device side coordination parallel execution models. Figure shows ops/watt achieved across increasing graph sizes.}
	\label{fig:energy_comp}
\end{figure}

\begin{figure}[t]
	\centering
	\subfloat[][bfs]{\includegraphics[width=0.3\textwidth]{io_reduction.pdf}\label{io-bfs}}
	\subfloat[][cc]{\includegraphics[width=0.3\textwidth]{io_reduction.pdf}\label{io-cc}}
	\subfloat[][pr-nibble]{\includegraphics[width=0.3\textwidth]{io_reduction.pdf}\label{io-pr}}
	\caption{This figure shows IO savings of device side coordination model over host side coordination model over increasing graph sizes.}
	\label{fig:ioreduction}
\end{figure}

\begin{figure}[t]
	\centering
	\subfloat[][bfs]{\includegraphics[width=0.3\textwidth]{graph-stalls.pdf}\label{stalls-bfs}}
	\subfloat[][cc]{\includegraphics[width=0.3\textwidth]{graph-stalls.pdf}\label{stalls-cc}}
	\subfloat[][pr-nibble]{\includegraphics[width=0.3\textwidth]{graph-stalls.pdf}\label{stalls-pr}}
	\caption{This figure shows the breakdown of stall cycles in the device side and host side coordination models over various benchmarks.}
	\label{fig:stalls}
\end{figure}



\subsection{Energy Results}
